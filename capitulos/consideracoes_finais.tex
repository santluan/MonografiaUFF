\chapter{Considerações Finais}

Nesta monografia foi possível apontarmos as diversas aplicações possíveis da análise de sentimentos nos variádos campos do conhecimento. No campo econômico, em particular, pudemos observar que o trabalho realizado ainda é recente, porém muito promissor. Vemos como através desta técnica pudemos desenvolver um índice de sentimentos que pudesse expressar os sentimentos ilustrados no noticiário econômico e, desse modo, da economia em geral. Isto é, pudemos observar o "humor" dos agentes econômicos frente as pespectívas atuais e futuras refletidas no nosso índice.

Logo em seguida, foram precisos realizar testes econométricos sob os índices para compreendermos a sua relevância econômica. Desse modo, para concretização deste exercício, pode ser verificado a natureza das séries sob a ótica dos testes de raiz unitária Dickey-Fuller Aumentado e Phillips-Perron. Os resultados dos testes demonstraram que os índices possuem comportamento estacionário, mais específicamente quando há presença de constante e tendência.

Seguindo assim a literatura de séries temporais, foi escolhido prosseguir com a estimação do modelo VAR para este exercício pela sua habilidade em tratar séries estacionários e endógenas. Por fim, com o modelo em mãos, foi estimado a Função de Resposta ao Impulso sobre as demais séries econômicas simulando um choque positivo causado pelos índices de sentimento. 

Seus resultados nos mostraram que o índice VADER (ISV), apesar de capturar bem as movimentações de mercado, não obteve desempenho semelhante nas demais séries econômicas. Isto é, quando observado os dados econômicos, não foi possível obter um entendimento claro para economia, sendo visto um comportamento contraditório, com níveis de atividade econômica positiva simultaneos a quedas acentuadas nos preços e taxa de juros com movimentos oscilantes.

Enquanto que, para o índice LM-SA (ISL), pudemos observar um bom desempenho em ambos, com um índice satisfatório ao refletir o humor econômico e também em previsionar movimentações nas variáveis. Ou seja, seus resultados puderam demonstrar maior coerência econômica quanto aos comportamentos das séries de dados, com altas na atividade coincidindo com elevações dos preços e juros no longo prazo. 

Tais resultados demonstraram que, para o meio econômico, o índice baseado no dicionário LM-SA é melhor frente ao VADER. Isto é explicado pela maneira de construção de cada um, sendo LM-SA um léxico com fundamentação econômica, nos demonstrou resultados mais realistas para economia.

Por fim, tendo isso dito, os resultados apresentados nesta monografia nos dão indícios de que a concepção de índices de sentimento de notícias tem relevância para compreensão dos fenômenos econômicos. Além disso, evidencia a importância do dicionário de termos que se apropriem ao campo de estudo no qual esta sendo realizada a pesquisa.
