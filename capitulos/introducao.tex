\chapter{Introdução}
\label{chap:introducao}

É notável o crescimento da ferramenta computacional da "análise de sentimentos"  (\textit{sentiment analysis}) e suas diversas aplicações para o entendimento e detecção de emoções implícitas em textos, livros, artigos, notícias entre outros tantos conteúdos escritos, mas principalmente no que se refere ao meio acadêmico, onde essa técnica ainda é relativamente recente, mas que tem ganhado grande espaço.

A análise de sentimentos é a técnica que se ocupa em detectar e mensurar a sensibilidade em conteúdo escrito, tendo vindo principalmente do campo dos negócios, sendo aplicadas ao marketing e serviços ao cliente, até chegar às análises de redes sociais e conteúdos de mídia e críticas especializadas, livros e qualquer outro conteúdo escrito pelos humanos. A partir disso, esse campo tem tido grandes avanços, principalmente nas ciências sociais, ao buscarem emoções num conteúdo de período específico ou encontrar sentimentos em bibliografias de autores. Além disso, a análise textual e de sentimentos também envolve outros meios como a da ciência da computação, através da mineração de dados e de textos em veículos de informação, da internet como um todo e também à capacidade de processamento que os computadores atuais podem processar. Não obstante, ela também envolve técnicas matemáticas e estatísticas para a realização de seus testes e medição de resultados.

Infelizmente esse campo de pesquisa aplicada à economia ainda é bem embrionário. Por enquanto existem poucas pesquisas a fim de extrair informações úteis de texto, também chamadas \textit{soft information}. Entretanto, dada a imensa quantidade de \textit{soft information} que são geradas diariamente, seja em jornais, relatórios financeiros, livros, redes sociais, se torna evidente a reavaliação desses meios como fontes novas de informação para entendimento dos fenômenos econômicos \cite{shap-sud}.

No meio econômico, os trabalhos que envolvem análise de sentimentos remetem a criação de índices macroeconômicos a partir de notícias de mercado e como elas podem impactar nas séries econômicas, como mostra \citeonline{shap-sud}, análise de artigos financeiros e sua correlação com o retorno de capitais de empresas de capital aberto, como apresentado por \citeonline{garcia}. Esses são somente alguns exemplos, mas que já demonstram a imensa possibilidade de aplicações que existem nesse campo.  

Em vista a isso, o interesse deste trabalho é explorar tais técnicas aplicadas ao campo econômico que busquem formas alternativas para explicar fenômenos econômicos e também compreensão da influência das \textit{soft information} nos mercados. Dentro disso, pretende-se realizar testes estatísticos entre o sentimento dos jornais e os dados econômicos, dentre eles o índice de atividade econômica, taxa de juros, índice de preços ao consumidor e retornos do mercado acionário.

Para este experimento teremos que fazer uso de \textit{web scraping} através da linguagem de programação \textit{Python}, sendo ele um método computacional que irá auxiliar na coleta dos dados que serão usados no projeto. O \textit{web scraping} é uma maneira popular de obtenção de dados estruturados e não estruturados da web de forma automatizada. Ela serve para facilitar a extração de dados em grande escala e os quais não teriam fácil acesso para \textit{download}, manipulação ou tratamento. Ele será empregado para raspar as notícias de mercado disponíveis na web e então tabuladas para realização da análise de sentimentos.

Portanto, para esta pesquisa, como dito anteriormente, serão utilizados procedimentos como os de \textit{text mining} e \textit{sentiment analysis} para a formulação do modelo e então aplicação aos dados históricos de notícias e dados econômicos. Por conta disso, por definição, as variáveis independentes (preditoras) e dependente (prevista) serão os termos e expressões em notícias e os dados econômicos, respectivamente. O meu intuito é então encontrar relações entre as variáveis e discorrer sobre seus resultados.

%O objetivo geral deste estudo é a análise de sentimentos das notícias de cunho econômico e financeiro e sua relação com o cenário econômico brasileiro. O estudo busca encontrar as relações entre os dois fatores, qual a sua magnitude, se existe alguma relação de causa e efeito ou apenas correlação mínima entre elas. Dentro disso, surge o objetivo específico desta pesquisa, que se centra em entender essa relação que os dois fatores possuem um ao outro, sendo as notícias servindo como parâmetro para a previsão de variações no quadro econômico.

No próximo capítulo serão encontrados os meios empregados para a coleta e tratamento de dados, assim como os trabalhos realizados nesse campo do conhecimento que inspiram essa monografia. Além disso, será discorrido sobre as técnicas de análise de sentimentos, especificidades contidas na área e meios para construção dos léxico de sentimentos usados neste trabalho.

Em seguida, no terceiro capítulo, será discorrido sobre a metodologia utilizada neste trabalho, além de realizar breve revisão dos métodos utilizados em econometria de séries temporais. No quarto capítulo, serão feitos a análise de dados e efetuação dos experimentos estatísticos para o objeto de estudo.

Finalmente, no encerramento será feito um apanhado dos resultados encontrados neste projeto de monografia, concluindo assim com a avaliação dos experimentos estatísticos e nossas considerações finais.

%Em seguida, em seu terceiro capítulo, a metodologia empregada neste trabalho será demonstrada em detalhes. Logo após, será feita a análise dos dados e realização dos experimentos estatísticos para o objeto de estudo. Por fim, no encerramento será feito um apanhado dos resultados encontrados neste projeto de monografia, concluindo assim com as suas considerações finais.

%será encontrada a explicação dos meios empregados para coleta e tratamento de dados, assim como a escolha da base de dados o qual será utilizada para o exercício desse projeto. Em seguida, demonstrará a metodologia da análise dos dados que será usada, em outras palavras, a análise de sentimentos será posta em exercício aos dados empíricos. Em seguida, o terceiro capítulo serão realizados os experimentos estatísticos no campo econômico, onde se esperam encontrar resultados relevantes quanto à relação entre os objetos de estudo. Por fim, no capítulo de encerramento, será trago um apanhado do estudo realizado nesse projeto de monografia, concluindo com as considerações finais.
