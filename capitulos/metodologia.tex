\chapter{Metodologia}

Nesta seção será realizada breve revisão dos métodos de econometria aplicadas às séries temporais. Vale lembrar que a disciplina da econometria ocupa o papel de medição estatística dentro da economia e compreensão dos fenômenos econômicos através dela. Enquanto que séries temporais trata-se da modelagem de variáveis observadas sequencialmente ao longo do tempo, sendo bem comum às áreas de finanças e economia. Tendo isso em mente, este capítulo cobrirá os principais tópicos dessas disciplinas utilizados no estudo desta monografia.

\section{Processos Estacionários}

Em séries temporais, descobrir o caráter da série a ser estudada é essencial, pois a partir dela é possível selecionar o modelo ideal para estimação. Neste caso, é preciso compreender se as variáveis são estacionárias. Segundo \citeonline{wooldridge}:

\begin{citacao}
"Um processo estacionário de série temporal é aquele em que as distribuições de probabilidades são estáveis no decorrer do tempo no seguinte sentido: se pegarmos qualquer coleção de variáveis aleatórias na sequência e depois deslocarmos essa sequência para diante em $h$ períodos de tempo, a distribuição de probabilidade conjunta deve permanecer inalterada"
\cite{wooldridge}
\end{citacao}

Em outras palavras, um processo estocástico é tido como estacionário se suas propriedades estatísticas se mantiverem inalteradas independente do intervalo de tempo que forem extraídas, ou seja, ao selecionarmos qualquer período de tempo com intervalo $h$ de distância entre as observações, sua propriedade estatística não se altera, permanecendo assim constante. Seguindo a sua definição formal, um processo estocástico é estacionário se:
\begin{equation}\label{stacionarity}
\begin{split}
    E[y_t] = \mu, \\
    Var[y_t] = \sigma^2_y, \\
    Cov[y_t, y_{t-s}] = \gamma_{(s)},
\end{split}
\end{equation}

\noindent onde $\mu$, $\sigma^2_y$ são constantes e $\gamma_{(s)}$ é uma função de $|s|$. Portanto, a Equação (\ref{stacionarity}) denota que para um processo ser considerado estacionário, é preciso que: (i) sua média condicional seja constante e finita; (ii) sua variância seja constante e finita; e (iii) suas auto-correlações dependerem apenas da distância entre $t$ e $t-s$ e forem constantes e finitas.

Um exemplo de processo estacionário é o chamado ruído branco $\varepsilon_t$ quando satisfaz as condições:

\begin{equation}\label{ruidobranco}
\begin{split}
    E[\varepsilon_t] = 0, \\
    Var[\varepsilon_t] = \sigma^2, \\
    Cov[\varepsilon_t, \varepsilon_{t-s}] = 0.
\end{split}
\end{equation}

Outro tipo de processo são os chamadados processos autorregressivos (AR), os quais expressam a variável aleatória no tempo $t$ como resultado da combinação linear das $k$ variáveis a anteriores. Assim sendo, um processo AR(1) adquire a forma 
\begin{equation}
    y_t = \rho y_{t-1} + \varepsilon_t
\end{equation}

\noindent onde $\varepsilon_t$ é ruído branco. Como monstrado em \citeonline{wooldridge} se $\rho$ = 1, o processão não é estacionário e, se $\rho$ < 1, o processo é estacionário.

\section{Testes de raiz unitária}

Na literatura de séries temporais, os testes de estacionariedade, também comumente conhecidos como \textit{Unit Root Test}, são procedimentos estatísticos para checar a presença de raiz unitária. Em outras palavras, os testes de raiz unitária verificam se existem evidências de estacionariedade na série.

%\begin{align*} \label{hipoteses}
%    H_O: \rho = 0 \\
%    H_A: \rho < 0
%\end{align*}

%\noindent onde, a hipótese nula indica que a série possui tendência estocástica, enquanto que a hipótese alternativa indica o contrário, não possui tendência. Ou seja, a interpretação do teste é dita como: ao rejeitarmos a hipótese nula, não há presença de tendência, que por consequência, não há presença de raiz unitária na série.

%Normalmente, o teste para examinar a presença de raíz unitária envolve dois passos: estimação de modelo auto-regressivo AR(p) que represente razoavelmente os dados e então testar a hipótese nula de raíz unitária.

\subsection{Teste Dickey-Fuller (DF)}

Por conta da dificuldade encontrada ao examinar séries de tempo, testes de estacionariedade foram desenvolvidos para esta tarefa, sendo os mais comuns o Teste Dickey-Fuller, Dickey-Fuller Aumentado e Phillips-Perron, todos eles sendo explicados a seguir. De acordo com \citeonline{principles-econometrics}, a definição do teste Dickey-Fuller (DF) pode ser expresso por um modelo AR(1) com diferentes circunstâncias.

%Inicialmente, o teste Dickey-Fuller (DF) envolve a estimação de uma modelo autorregressivo de ordem 1, AR(1). Em seguida, são consideradas três expressões para teste de raiz unitária, sendo elas:

%\begin{equation}\label{DF}
%\begin{split}
%    \Delta{y_t} = \rho y_{t-1} + \varepsilon_t\quad, \\
%    \Delta{y_t} = \alpha_0 + \rho y_{t-1} + \varepsilon_t\quad, \\
%    \Delta{y_t} = \alpha_0 + \rho y_{t-1} + \alpha_2{t} + \varepsilon_t\quad.
%\end{split}
%\end{equation}

\begin{table}[!h]
    \centering
    \caption{Processos AR e testes Dickey-Fuller}
    \begin{tabular}{ccccc}
    \hline
Processos AR: $|\rho|$ < 1 && $\rho$ = 1 && Teste Dickey-Fuller \\
    \hline \\ \hline
$y_t = \rho y_{t-1} + \varepsilon_t$ && $y_t = y_{t=1} + \varepsilon_t$ && Teste sem constante e sem tendência \\
    \hline \\ \hline
$y_t = \alpha + \rho y_{t-1} + v_t$ && $y_t = y_{t=1} + v_t$ && Teste com constante e sem tendência \\
$\alpha = \mu(1-\rho)$ && $\alpha=0$ &&      \\
    \hline \\ \hline
$y_t = \alpha + \rho y_{t-1} + \lambda t + v_t$ && $y_t = \delta + y_{t-1} + v_t$ && Teste com constante e tendência \\
$\alpha = \mu(1-\rho) + \rho \delta$ && $\alpha = \delta$ &&     \\
$\lambda = \delta(1-\rho)$ && $\lambda = 0$      \\
    \hline
    \end{tabular} \label{DFtests}
    \fonte{\cite{principles-econometrics}}
\end{table}

A Tabela \ref{DFtests} demonstra o teste DF nas suas três formas, sendo: (i) presença de passeio aleatório; (ii) adição de constante; e (iii) presença de um elemento de tendência e constante. Sua interpretação é dada por $\rho$. 

%Se $\rho$ , o processo contém raiz unitária. 

Por fim, o teste envolve a estimação de equações usando Mínimos Quadrados Ordinários (MQO) para se obter o valor estimado de $\rho$. Uma vez obtido o resultado, basta comparar sua estatística $t$ com as tabelas DF para determinar a aceitação ou rejeição da hipótese nula, ou mais facilmente a interpretação do seu $p$-valor. Para os fins dessa monografia, foram realizados o cálculo para o teste através do pacote econométrico aTSA (\citeonline{aTSA}) do software R (\citeonline{R}).

\subsection{Teste Dickey-Fuller Aumentado (ADF)}

A diferença entre o teste Dickey-Fuller Aumentado e o teste DF simples, está na sua ordem de estimação. Segundo \citeonline{enders}, a metodologia do teste DF é estimada por um processo autorregressivo de ordem 1, AR(1), enquanto que o teste ADF é estimado um processo autoregressivo de ordem $\rho$, AR($\rho$), com tantas defasagens forem necessárias para diminuir os efeitos residuais da série. Essa evolução vem para melhor representar o caráter das séries temporais não-estacionárias. Dessa forma, a metodologia do teste admite uma representação AR($\rho$): 
\begin{equation}\label{ADF1}
    y_t = \alpha_0 + \alpha_1 y_{t-1} + \alpha_2 y_{t-2} + ... + \alpha_{p-1} y_{t-p+1} + \alpha_p y_{t-p} + \varepsilon_t\quad,
\end{equation}

\noindent que a partir da Equação (\ref{ADF1}), adiciona-se e subtrai-se $\alpha_p y_{t-p+1}$ para obter
\begin{equation}\label{ADF2}
    y_t = \alpha_0 + \alpha_1 y_{t-1} + \alpha_2 y_{t-2} + ... + (\alpha_{p-1} + \alpha_p) y_{t-p+1} - \alpha_p \Delta y_{t-p+1} + \varepsilon_t \quad,
\end{equation}

\noindent continuando neste método de subtrair-se $(\alpha_{p-1} + \alpha_p)y_{t-p+2}$, até se chegar a equação
\begin{equation}\label{ADF3}
    \Delta y_t = \alpha_0 + \gamma y_{t-1} + \sum^{p}_{i=2} \beta_i \Delta y_{t-i+1} + \varepsilon_t \quad,
\end{equation}

\noindent onde $\gamma = - (1 - \sum^p_{i=1} \alpha_i)$ e $\beta_i = \sum^{p}_{j=i} \alpha_j$.

Como explicado anteriormente no teste DF, nosso termo de interesse aqui está no coeficiente $\gamma$ exemplificado na Equação (\ref{ADF3}). Sendo $\gamma$ = 0, o processo dispõe de raiz unitária, o que acarreta em não estacionariedade. As hipóteses do teste são apresentadas a seguir:
\begin{align*} \label{hipoteses}
    H_O: \gamma = 0 \\
    H_A: \gamma < 0
\end{align*}

\noindent sendo $H_0$ e $H_A$ as hipóteses nula e hipótese alternativa, respectivamente.

Além disso, como discutido antes, para consultar seu resultado basta comparar sua estatística-$t$ com as estatísticas tabuladas por Dickey-Fuller. Para mais detalhes, consultar \citeonline{enders}.

\subsection{Teste Philips-Perron (PP)}

Como explicado por \citeonline{bueno}, o teste desenvolvido por Phillips e Perron (1988) surge a partir dos preceitos definidos pelo teste ADF, porém, nele é incluído uma correção não-paramétrica que permite ao estimador seja consistente para variáveis defasadas dependentes e contenham correlação serial nos erros. O que, portanto, torna desnecessária a estimação de modelo autorregressivo com tantas defasagens suficientes para mitigar os efeitos residuais contidos na série, como é feito no teste ADF.

Dessa maneira, considerando que seus procedimentos são análogos aos do teste ADF, tanto na estimação e teste para hipótese nula, a interpretação do modelo também é equivalente. Portanto, a equação de teste PP pode ser descrita como:
\begin{equation}
    \begin{split}
        \Delta y_t = \alpha y_{t-1} + \varepsilon_t \quad, \\
        \Delta y_t = \mu + \alpha y_{t-1} + \varepsilon_t \quad, \\
        \Delta y_t = \mu + \delta t + \alpha y_{t-1} + \varepsilon_t \quad,
    \end{split}
\end{equation}

\noindent sendo $\varepsilon_t$ um ruído branco.


\section{Modelo Vetor Autorregressivo - VAR}

Idealizado por \citeonline{sims}, o modelo vetor autorregressivo vem para suprir a necessidade das ciências econômicas no estudo de modelos estimados com múltiplas variáveis econômicas.

\begin{citacao}
"Os modelos econômicos em geral são expressos por meio de diversas variáveis. Portanto, o uso de modelos univariados é limitado para expressar modelos econômicos. O vetor autorregressivo permite que se expressem modelos econômicos completos e se estimem os parâmetros desse modelo."
\cite{bueno}
\end{citacao}

Ou seja, diferentemente das restrições impostas pelos modelos univariados, o modelo VAR permite que modelos econômicos sejam expressos na sua forma completa, possibilitando assim a análise do comportamento conjunto das variáveis ao tratar todas como endógenas.

Desse modo, o modelo autorregressivo vetorial pode ser ser expresso pela seguinte equação extraída de \citeonline{bueno}:

\begin{equation}\label{var-estrutural}
    AX_t = B_0 + \sum^p_{i=1} B_i X_{t-i} + B\varepsilon_t \quad,
\end{equation}

\noindent onde $X_t$ representa o vetor de $n$ processos endógenos, $n \times 1$; conectadas a ele está $\mathbf{A}$, uma matrix $n \times n$ que compõe as restrições das variáveis comtemporaneamente; $\mathbf{B_0}$ é um vetor de constantes $n \times 1$; $\mathbf{B_i}$ são matrizes $n \times n$; $\mathbf{B}$ é uma matriz diagonal $n \times n$ e $\varepsilon_t$ um vetor $n \times 1$ de termo de erro aleatório não correlacionados entre si temporalmente.

A Equação (\ref{var-estrutural}) representa a forma estrutural no modelo VAR. Entretanto, um modelo econômico teórico não pode ser estimado em sua forma estrutural, pois existe correlação contemporênica entre os regressos e os choques, por conta de sua natureza endógena, gerando assim estimadores inconsistentes. 

Por esses motivos, o modelo VAR é estimado na sua forma reduzida, podendo ser representado pela equação a seguir:
\begin{align}
\begin{split}
    X_t = A^{-1}B_0 + \sum^p_{i=1} A^{-1} B_i X_{t-1} + A^{-1} B\varepsilon_t \quad, \\
    X_t = \Phi_0 + \sum^p_{i=1} \Phi_i X_{t-1} + \epsilon_t \quad,
\end{split}
\end{align}

\noindent em que $\Phi \equiv \mathbf{A^{-1}}B_i$, $i$ = 0,1,...,p e $B\varepsilon_t \equiv A\epsilon_t$.

%O sua forma reduzida pode ser explicada por meio de um modelo bivariado de ordem 1, exemplificado por \cite{enders}. 
A Equação (\ref{var-reduzido}) contém a simplificação do modelo estrutural e nos permite visualizar mais facilmente o comportamento contemporâneo das séries temporais. É possível observar como os dois processos se influenciam mutuamente e contemporaneamente \cite{enders}.
\begin{align}\label{var-reduzido}
    \begin{split}
    y_t = b_{10} - b_{12}z_t + \gamma_{11}y_{t-1} + \gamma_{12}z_{t-1} + \epsilon_{yt} \\
    z_t = b_{20} - b_{21}y_t + \gamma_{21}y_{t-1} + \gamma_{22}z_{t-1} + \epsilon_{zt}
    \end{split}
\end{align}

Por fim, o modelo vetorial autorregressivo objetiva o desenvolvimento de técnicas que diminuam o efeito retroativo entre os termos de erros das séries e assim estimar a tragetória do processo de interesse ante um choque nos erros, isto é, um choque estrutural \cite{bueno}.

\subsection{Função Resposta ao Impulso}

A função resposta ao impulso é uma ferramenta metodológica formalizada por \citeonline{sims} visando entender o comportamento das séries dado um choque nas variáveis do modelo VAR. Em outras palavras, a função resposta impulso procura rastrear o comportamento futuro das séries do modelo dado um choque estrutural em seus termos de erro. 

Assim como um o modelo AR($\rho$) possui uma representação de média móvel, o vetor autorregressivo também pode ser escrito como um vetor de média móvel (VMA). Em termos matemáticos, a representação do VMA é expressa pelo seu valor corrente e valores passados do modelo  e permite traçar o caminho temporal dos vários choques nas séries contidas no sistema VAR. Para ilustrar esse método, iremos seguir a forma matricial do VAR(1) para duas variáveis exemplificada por \citeonline{enders}:
\begin{align} \label{var-mat1}
\begin{bmatrix}
    y_t \\
    z_t
\end{bmatrix}
=
\begin{bmatrix}
    a_{10} \\
    a_{10} \\
\end{bmatrix}
+
\begin{bmatrix}
    a_{11} & a_{12} \\
    a_{21} & a_{22} 
\end{bmatrix}
\begin{bmatrix}
    y_{t-1} \\
    z_{t-1} \\
\end{bmatrix}
+
\begin{bmatrix}
    \epsilon_{1t} \\
    \epsilon_{2t} \\
\end{bmatrix}    
\end{align}

\noindent podendo ser reescrita como
\begin{align} \label{var-mat2}
\begin{bmatrix}
    y_t \\
    z_t
\end{bmatrix}
=
\begin{bmatrix}
    \overline{y} \\
    \overline{z}
\end{bmatrix}
+
\sum_{i=0}^{\infty}
\begin{bmatrix}
    a_{11} & a_{12} \\
    a_{21} & a_{22} 
\end{bmatrix}^i
\begin{bmatrix}
    \epsilon_{1t-i} \\
    \epsilon_{2t-i}
\end{bmatrix} \quad,
\end{align}

\noindent vale salientar que a matriz $\begin{bmatrix} a_{11} & a_{12} \\ a_{21} & a_{22} \end{bmatrix}$ é resultado de um processo de inversão dos operadores definida pela matriz $\begin{bmatrix} \gamma_{11} & \gamma_{12} \\ \gamma_{21} & \gamma_{22} \end{bmatrix}$. Detalhes podem ser consultados na referência \cite{enders}.

A Equação (\ref{var-mat2}) expressa $y_t$ e $z_t$ em termos de {$\epsilon_{1t}$} e {$\epsilon_{2t}$}. Entretanto, ao reescrevermos a equação em termos de $\varepsilon_{1t}$ e $\varepsilon_{2t}$, teremos o vetor de erros descrito como
\begin{align}\label{var-erro}
\begin{bmatrix}
    \epsilon_{1t} \\
    \epsilon_{2t}
\end{bmatrix}
=
\frac{1}{1 - b_{12}b_{21}}
\begin{bmatrix}
    1       & -b_{12} \\
    -b_{21} & 1
\end{bmatrix}
\begin{bmatrix}
    \varepsilon_{yt} \\
    \varepsilon_{zt}
\end{bmatrix} \quad,
\end{align}

\noindent substituindo (\ref{var-erro}) em (\ref{var-mat2}) segue
\begin{align} \label{var-mat3}
\begin{bmatrix}
    y_t \\
    z_t
\end{bmatrix}
=
\begin{bmatrix}
    \overline{y} \\
    \overline{z}
\end{bmatrix}
+
\frac{1}{1 - b_{12}b_{21}}
\sum_{i=0}^{\infty}
\begin{bmatrix}
    a_{11} & a_{12} \\
    a_{21} & a_{22} 
\end{bmatrix}^i
\begin{bmatrix}
    1       & -b_{12} \\
    -b_{21} & 1
\end{bmatrix}
\begin{bmatrix}
    \varepsilon_{yt-i} \\
    \varepsilon_{zt-i}
\end{bmatrix} \quad.
\end{align}

Dada sua imensa notação, pode-se simplificar a Equação (\ref{var-mat3}) definindo o termo $\phi$ como uma matrix $2 \times 2$ com elementos $\phi_{jk}(i)$
\begin{align}
\phi_i =
\frac{A_{1}^i}{1 - b_{12}b_{21}}
\begin{bmatrix}
    1       & -b_{12} \\
    -b_{21} & 1
\end{bmatrix} \quad,
\end{align}

\noindent assim temos a representação do VMA em termos de {$\varepsilon_{yt}$} e {$\varepsilon_{zt}$}, como seguem:
\begin{align}\label{vma-final}
\begin{bmatrix}
    y_t \\
    z_t
\end{bmatrix}
=
\begin{bmatrix}
    \overline{y} \\
    \overline{z}
\end{bmatrix}
+
\sum_{i=0}^{\infty}
\begin{bmatrix}
    \phi_{11}(i) & \phi_{12}(i) \\
    \phi_{21}(i) & \phi_{22}(i)
\end{bmatrix}
\begin{bmatrix}
    \varepsilon_{yt-i} \\
    \varepsilon_{zt-i}
\end{bmatrix} \quad,
\end{align}

\noindent ou na sua sua forma simplificada
\begin{align} \label{VAR-FIR}
    x_t = \mu + \sum_{i=0}^{\infty} \phi_i \varepsilon_{t-i}
\end{align}

\noindent onde $x_t = \begin{bmatrix} y_t \\ z_t \end{bmatrix}$; $\mu = \begin{bmatrix} \overline{y} \\ \overline{z} \end{bmatrix}$; $\phi_i = \begin{bmatrix} \phi_{11}(i) & \phi_{12}(i) \\ \phi_{21}(i) & \phi_{22}(i) \end{bmatrix}$ e $\varepsilon_{t-i} = \begin{bmatrix} \varepsilon_{yt-i} \\ \varepsilon_{zt-i} \end{bmatrix}$.

Essa representação do VMA é um artifício muito útil para examinar a interação entre as variáveis {$y_t$} e {$z_t$}. Essencialmente, os coeficientes $\phi_i$ podem ser utilizados para gerar choques de $\varepsilon_{yt}$ e $\varepsilon_{zt}$ nas sequências de {$y_t$} e {$z_t$}. Em outras palavras, denota-se que coeficientes são os principais causadores do choque nos erros e, por isso, são chamados de multiplicados de impacto.

Para se chegar ao efeito acumulado na série, basta somar os impulsos unitários de $\phi_i$ em $\varepsilon_{yt}$ e $\varepsilon_{zt}$ para se obter a função de impulso.
\begin{align}
    \sum_{i=0}^{\infty} \phi_{jk}(i)
\end{align}

Dessa forma, os coeficientes $\phi_{11}(i)$, $\phi_{12}(i)$, $\phi_{21}(i)$ e $\phi_{22}(i)$ da Equação (\ref{vma-final}) são chamados de Função Resposta ao Impulso.
