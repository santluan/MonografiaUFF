\begin{resumo}

É evidente o aumento de alternativas viáveis para o estudo e entendimento dos fenômenos econômicos, desde saber o seu comportamento futuro, antecipar acontecimentos e previsionar movimentações de mercado. Como também ficou claro as diversas aplicações que os avanços dos métodos computacionais e quantitativos proporcionaram para esta análise nos anos recentes, com exemplos da análise de sentimentos e mineração de dados. Tendo isso em vista, essa monografia propõe o emprego da análise de sentimentos às notícias econômicas e financeiras na busca do entendimento do seu impacto sobre a economia e agentes econômicos. Para realização deste exercício também serão empregadas técnicas de \textit{web scraping} para coleta de dados, além da aplicação de técnincas econométricas de vetores autoregressivos. Por fim, o entendimento deste fato é de extrema importância para identificar oportunidades e explorar novas estratégias e visões de mercado.


\textbf{Palavras-chave}: Análise de Sentimentos; Notícias; Modelo Vetor Autorregressivo; Função Resposta Impulso.

\end{resumo}
